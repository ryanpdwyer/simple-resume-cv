% !TEX TS-program = xelatex
% !TEX encoding = UTF-8 Unicode
% -*- coding: UTF-8; -*-
% vim: set fenc=utf-8

%%%%%%%%%%%%%%%%%%%%%%%%%%%%%%%%%%%%%%%%%%%%%%%%%%%%%%%%%%%%%%%%%
%% SIMPLE-RESUME-CV
%% <https://github.com/zachscrivena/simple-resume-cv>
%% This is free and unencumbered software released into the
%% public domain; see <http://unlicense.org> for details.
%%%%%%%%%%%%%%%%%%%%%%%%%%%%%%%%%%%%%%%%%%%%%%%%%%%%%%%%%%%%%%%%%

%%%%%%%%%%%%%%%%%%%%%%%%%%%%%%%%%%%%%%%%%%%%%%%%%%%%%%%%%%%%%%%%%
%% INSTRUCTIONS FOR COMPILING THIS DOCUMENT ("CV.tex")
%% TeX ---(XeLaTeX)---> PDF:
%%
%% Method 1: Use latexmk for fully automated document generation:
%%   latexmk -xelatex "CV.tex"
%%   (add the -pvc switch to automatically recompile on changes)
%%
%% Method 2: Use XeLaTeX directly:
%%   xelatex "CV.tex"
%%   (run multiple times to resolve cross-references if needed)
%%%%%%%%%%%%%%%%%%%%%%%%%%%%%%%%%%%%%%%%%%%%%%%%%%%%%%%%%%%%%%%%%

\documentclass[letterpaper,MMMyyyy,nonstop]{simpleresumecv}
% Class options:
% a4paper, letterpaper, draft, nonstop
% MMMyyyy, ddMMMyyyy, MMMMyyyy, ddMMMMyyyy, yyyyMMdd, yyyyMM, yyyy

%%%%%%%%%%%%%%%%%%%%%%%%%%%%%%%%%%%%%%%%%%%%%%%%%%%%%%%%%%%%%%%%%
%% PREAMBLE.
%%%%%%%%%%%%%%%%%%%%%%%%%%%%%%%%%%%%%%%%%%%%%%%%%%%%%%%%%%%%%%%%%

% CV Info (to be customized).
\newcommand{\CVAuthor}{Ryan Dwyer}
\newcommand{\CVTitle}{}
\newcommand{\CVNote}{}
\newcommand{\CVWebpage}{ryanpdwyer.github.io}

% PDF settings and properties.
\hypersetup{
pdftitle={\CVTitle},
pdfauthor={\CVAuthor},
pdfsubject={\CVWebpage},
pdfcreator={XeLaTeX},
pdfproducer={},
pdfkeywords={},
pdfpagemode={},
unicode=true,
bookmarks=true,
bookmarksopen=true,
pdfstartview=FitH,
pdfpagelayout=OneColumn,
pdfpagemode=UseOutlines,
hidelinks,
breaklinks}

% Shorthand.
\newcommand{\CodeCommand}[1]{\mbox{\textbf{\textbackslash{#1}}}}

%%%%%%%%%%%%%%%%%%%%%%%%%%%%%%%%%%%%%%%%%%%%%%%%%%%%%%%%%%%%%%%%%
%% ACTUAL DOCUMENT.
%%%%%%%%%%%%%%%%%%%%%%%%%%%%%%%%%%%%%%%%%%%%%%%%%%%%%%%%%%%%%%%%%

\begin{document}

%%%%%%%%%%%%%%%
% TITLE BLOCK %
%%%%%%%%%%%%%%%

\title{\CVAuthor}

\begin{subtitle}
150 Baker Lab, Ithaca, New York 14853, USA
\par
\href{mailto:rpd78@cornell.edu}
{rpd78@cornell.edu}
\,\SubBulletSymbol\,
+1\,(262)\,506-8712
\,\SubBulletSymbol\,
\href{\CVWebpage}
{\CVWebpage}
\end{subtitle}

\begin{body}

%%%%%%%%%%%%%%%
%% EDUCATION %%
%%%%%%%%%%%%%%%

\section
{Education}
{Education}
{PDF:Education}

\href{http://www.cornell.edu}
{\textbf{Cornell Univeristy}},
Ithaca, New York

\GapNoBreak
\BulletItem
Doctor of Philosophy (Ph.D.) in
\href{http://chem.cornell.edu}
{Chemistry}, 3.8/4.0
\hfill
\DatestampYM{2011}{08} --
\DatestampYM{2017}{05}
\begin{detail}
\SubBulletItem
Adviser:
Prof.~John Marohn
\SubBulletItem
Focus:
Scanned probe microscopy, organic semiconductors, data analysis and modeling
\end{detail}

\BigGap
\href{https://www.nd.edu}
{\textbf{University of Notre Dame}},
South Bend, Indiana

\GapNoBreak
\BulletItem
Bachelor of Science (B.S.) in
\href{http://chemistry.nd.edu}
{Chemistry}, \emph{magna cum laude}, 3.9/4.0
\hfill
\DatestampYM{2007}{08} --
\DatestampYM{2011}{05}


%%%%%%%%%%%%
%% SKILLS %%
%%%%%%%%%%%%

\section
{Skills}
{Skills}
{PDF:Skills}
% Reframe this as:

% Home-built instrumentation and new experimental methods
% 


% Numerical computing: Signal processing, Bayesian and maximum likelihood data analysis
% Signal processing

% Modeling and differential equations

% Nanofabrication
% Fabricated transistor substrates using photolithography
% Trained on over 20 tools (lithography, etching, evaporation, sputtering, furnaces, etc)

% General scripting, 


\textbf{Python}: numerical computing, general scripting
\BulletItem Signal processing modules to analyze data from new scanned probe microscopy experiments
\BulletItem \href{http://github.com/ryanpdwyer/newtex}{newtex}, a utility for setting up a git repository on Dropbox, used for Marohn group papers and grants
\BulletItem Contributed a new feature to \href{http://github.com/hgrecco/pint}{Pint}, a popular units package

\textbf{Bayesian and maximum likelihood data analysis}
\BulletItem Bayesian analysis improved inference from poorly conditioned microscopy data
\BulletItem Eliminated systematic bias in cantilever characterization using  maximum likelihood model and Bayesian inference

\textbf{Electronics}
\BulletItem Design, layout (Eagle) and assembly of analog and digital circuits used for scanned probe microscopy
\BulletItem Microcontroller-based (Teensy) potentiostat, with control and plotting in Python (\href{http://github.com/ryanpdwyer/teensyio}{teensyio})

\textbf{Other programming: LabVIEW, Mathematica, Julia, C}
\BulletItem NI DAQ-based custom scanned probe microscopy experiments implemented in LabVIEW
\BulletItem Contributed bugfix to Julia ordinary differential equations (ODE) package

\textbf{Other:} Adobe Illustrator, EAGLE, LTSpice, Autodesk Inventor, Word, Excel, PowerPoint 

%%%%%%%%%%%%%%%%%%%%%%%%%
%% RESEARCH EXPERIENCE %%
%%%%%%%%%%%%%%%%%%%%%%%%%

\section
{Research Experience}
{Research Experience}
{PDF:ResearchExperience}

\href{http://chem.cornell.edu/marohn}
{\textbf{Cornell University, with Dr. John Marohn}}

\href{http://www3.nd.edu/~kamatlab/}{\textbf{University of Notre Dame, with Dr. Prashant V. Kamat}}

\BulletItem Bound copper (I) sulfide to reduced graphene oxide to improve performance in quantum dot solar cells using polysulfide electrolyte (\href{http://dx.doi.org/10.1021/jz201064k}{Radich, \emph{Dwyer}, Kamat, \emph{J. Phys. Chem. Lett.}, 2011, 2, 2453})


%%%%%%%%%%%%%%%%%%
%% PUBLICATIONS %%
%%%%%%%%%%%%%%%%%%

% \section
% {Publications}
% {Publications}
% {PDF:Publications}

% \subsection
% {Journals}
% {Journals}
% {PDF:Journals}

% \GapNoBreak
% \NumberedItem{[11]}
% \href{http://www.example.com/my-paper-doi-5}
% {\emph{J.~Doe}, J.~Citizen, and A.~Yone,
% ``On lasers and climate change,''
% \textit{Journal of Science},
% vol.~89,
% no.~2,
% pp.~4123--4133,
% \DatestampYM{2008}{02}.}

% % Note the use of {\CharSpace} for aligning shorter numbers.
% \Gap
% \NumberedItem{{\CharSpace}[1]}
% \href{http://www.example.com/my-paper-doi-4}
% {\underline{J.~Doe} and J.~Citizen,
% ``Measuring the extent of climate change,''
% \textit{Global Scientific Journal},
% vol.~12,
% no.~4,
% pp.~330--352,
% \DatestampYM{2006}{12}.}

% \BigGap
% \subsection
% {Conferences}
% {Conferences}
% {PDF:Conferences}

% \GapNoBreak
% \NumberedItem{[11]}
% \href{http://www.example.com/my-paper-doi-3}
% {\underline{J.~Doe}, J.~Citizen, and A.~Yone,
% ``On lasers and climate change,''
% in \textit{Proceedings of the Laser Symposium},
% Las Vegas, Nevada, USA,
% \DatestampYM{2007}{01}.}

% \Gap
% \NumberedItem{[10]}
% \href{http://www.example.com/my-paper-doi-2}
% {A.~Yone and \underline{J.~Doe},
% ``Climate change and general relativity,''
% in \textit{Proceedings of the International Astronomical Conference},
% Sydney, Australia,
% \DatestampYM{2006}{8}.}

% % Note the use of {\CharSpace} for aligning shorter numbers.
% \Gap
% \NumberedItem{{\CharSpace}[1]}
% \href{http://www.example.com/my-paper-doi-1}
% {\underline{J.~Doe} and J.~Citizen,
% ``Measuring the extent of climate change,''
% in \textit{Proceedings of the International Climate Change Conference},
% London, UK,
% \DatestampYM{2005}{11}.}

%%%%%%%%%%%%%%%%%%%%%%%%%%%
%% AWARDS & SCHOLARSHIPS %%
%%%%%%%%%%%%%%%%%%%%%%%%%%%

\section
{Honors}
{Honors}
{PDF:Honors}

Teaching Assistant Award, Cornell Department of Chemistry
\hfill
\DatestampY{2012}

University of Notre Dame Chemistry Research Award
\hfill
\DatestampY{2011}

University of Notre Dame William R. Wischerath Outstanding Major Award
\hfill
\DatestampY{2011}

Dean's List 7 semesters, University of Notre Dame
\hfill
\DatestampY{2007} --
\DatestampY{2011}

%%%%%%%%%%%%%%%%%%%%%%%%%%%%%%%%%%%%%%%%%%%%
%% PROFESSIONAL AFFILIATIONS & ACTIVITIES %%
%%%%%%%%%%%%%%%%%%%%%%%%%%%%%%%%%%%%%%%%%%%%

% \section
% {Professional Affiliations \newline
% \& Activities}
% {Professional Affiliations \& Activities}
% {PDF:ProfessionalAffiliationsActivities}

% \href{http://www.example.com/my-society}
% {\textbf{Society of Professional Earth Scientists}},
% New York, USA

% \GapNoBreak
% \BulletItem
% Member
% \hfill
% \DatestampY{2009} --
% Present

%%%%%%%%%%%%%%%%%%%%%%%
%% CAMPUS ACTIVITIES %%
%%%%%%%%%%%%%%%%%%%%%%%

\section
{Teaching and Outreach}
{Teaching and Outreach}
{PDF:TeachingOutreach}

\href{https://www.eyh.cornell.edu}
{\textbf{Expanding Your Horizons workshop}},
Cornell University
\hfill
\DatestampY{2013} --
\DatestampY{2017}

\BulletItem Created and led workshop where 9th grade girls build a wind generator to power LEDs 

\textbf{NSF GK12 Grassroots Fellow}
\hfill
\DatestampYM{2013}{06} -- \DatestampYM{2014}{06}
\BulletItem Supervised research of a high school teacher in the Marohn Group
\BulletItem Developed renewable energy related modules for high scool classes with other fellows and high school teachers
\BulletItem Presented classroom activities related to renewable energy and my research 

\textbf{Graduate Teaching Assistant Fellow}, Cornell Center for Teaching Excellence
\hfill
\DatestampYM{2012}{08}--\DatestampYM{2014}{06}

\BulletItem Led 12 workshops on teaching-related topics, assisted with 4 teaching symposia
\BulletItem Led international TA training sessions, providing feedback on teaching and communication


%%%%%%%%%%%%%%%%%%%%%%%%%%%
%% OTHER WORK EXPERIENCE %%
%%%%%%%%%%%%%%%%%%%%%%%%%%%
\Hide{
\section
{Other Work\newline
Experience}
{Other Work Experience}
{PDF:OtherWorkExperience}

{\bf Teacher Assistant,} Early Childhood Development Center,
Notre Dame, IN
\hfill
\DatestampY{2009} --
\DatestampY{2011}

\BulletItem Assisted teachers in the 4/5 classroom

\Gap

{\bf Camp Counselor,} Camp Helen Brachman, Almond, WI \hfill \DatestampY{2009}

\BulletItem 8/9 year old boys’ counselor, led biking and basketball activities

\Gap
{\bf Tutor,} University of Notre Dame, Notre Dame, IN \hfill \DatestampY{2008} -- \DatestampY{2009}

\BulletItem Ran chemistry student athlete study hall. Tutored chemistry, physics and mathematics individually
}


% %%%%%%%%%%%%%%%%%%%%%%%%%%%%%%%%%
% %% SECTION WITH USAGE EXAMPLES %%
% %%%%%%%%%%%%%%%%%%%%%%%%%%%%%%%%%

% \section
% {Section\newline
% With\newline
% Usage\newline
% Examples}
% {Section With Usage Examples (For PDF Bookmark)}
% {PDF:SectionWithUsageExamples:ForPDFLink}

% \subsection
% {This is a Subsection}
% {This is a Subsection}
% {PDF:ThisIsASubSection}

% \GapNoBreak
% \BulletItem
% Use \CodeCommand{section} and \CodeCommand{subsection} to create sections and subsections.
% These will appear in the PDF bookmarks too.

% \GapNoBreak
% \BulletItem
% This is the second \CodeCommand{BulletItem}.
% Long items are automatically indented.
% Lorem ipsum dolor sit amet, consectetur adipiscing elit.
% Sed sed aliquam massa.
% \begin{detail}
% \SubBulletItem
% This is a \CodeCommand{SubBulletItem}.
% Long items are automatically indented.
% Lorem ipsum dolor sit amet, consectetur adipiscing elit.
% Sed sed aliquam massa.
% Aliquam dignissim mi non enim feugiat elementum.
% Donec sit amet turpis ac velit ultrices volutpat.
% Aliquam vitae elit massa.
% \SubBulletItem
% This is the second \CodeCommand{SubBulletItem}.
% \SubBulletItem
% The \CodeCommand{SubBulletItem}'s are between
% \CodeCommand{begin\{detail\}} and
% \CodeCommand{end\{detail\}} so that they are typeset in a smaller font.
% \end{detail}

% \Gap
% \BulletItem
% This is the third \CodeCommand{BulletItem}.

% \Gap
% \BulletItem
% A \CodeCommand{Gap} or \CodeCommand{GapNoBreak} is inserted between the \CodeCommand{BulletItem}'s so that there is a small vertical space between them.
% The ``NoBreak'' version prevents page breaking, and should be used to avoid orphaned headings and other formatting issues.

% \BigGap
% \subsection
% {This is the Second Subsection}
% {This is the Second Subsection}
% {PDF:ThisIsTheSecondSubSection}

% \GapNoBreak
% \BulletItem
% A \CodeCommand{BigGap} or \CodeCommand{BigGapNoBreak} is inserted between subsections so that there is a bigger vertical space between them.
% The ``NoBreak'' version prevents page breaking.

% % %%%%%%%%%%%%%%%%%%%%%%%%%%%%%%%%%%%%%%%%%
% %% ANOTHER SECTION WITH USAGE EXAMPLES %%
% %%%%%%%%%%%%%%%%%%%%%%%%%%%%%%%%%%%%%%%%%

% \section
% {Another\newline
% Section\newline
% With\newline
% Usage\newline
% Examples}
% {Another Section With Usage Examples (For PDF Bookmark)}
% {PDF:AnotherSectionWithUsageExamples:ForPDFLink}

% \textbf{This is a Plain Heading},
% followed by an \CodeCommand{hfill} and a date range
% \hfill
% \DatestampYM{2015}{10} --
% \DatestampYM{2015}{12}

% \GapNoBreak
% \BulletItem
% This is a \CodeCommand{BulletItem}.
% \begin{detail}
% \SubBulletItem
% This is a \CodeCommand{SubBulletItem}.
% \end{detail}

% \GapNoBreak
% \BulletItem
% This is a \CodeCommand{BulletItem}.
% \begin{detail}
% \SubItem
% This is a \CodeCommand{SubItem}, which has no bullet.
% Note the alignment with the \CodeCommand{BulletItem} above.
% \end{detail}

% \GapNoBreak
% \Item
% This is an \CodeCommand{Item}, which has no bullet.
% Note the alignment with the \CodeCommand{BulletItem} above.
% \begin{detail}
% \SubItem
% This is a \CodeCommand{SubItem}.
% \end{detail}

% \GapNoBreak
% \NumberedItem{[16]}
% This is a \CodeCommand{NumberedItem}.
% Note the alignment with the \CodeCommand{SubBulletItem} above.

% \GapNoBreak
% \NumberedItem{{\CharSpace}[6]}
% This is a \CodeCommand{NumberedItem} with a \CodeCommand{CharSpace} in its argument for padding shorter numbers.
% Note the alignment with the \CodeCommand{NumberedItem} above.

% \textbf{Usage Notes}

% \GapNoBreak
% \BulletItem
% New Lines and Paragraphs
% \begin{detail}
% \SubBulletItem
% To create a new line within the same paragraph (i.e., with the same indentation), use \CodeCommand{newline} instead of \CodeCommand{\textbackslash}.
% The latter will not work because it breaks the long table.
% \SubBulletItem
% To create a new paragraph, use \CodeCommand{par} or simply leave an empty line.
% Paragraph indentations (from
% \CodeCommand{Item},
% \CodeCommand{SubItem},
% \CodeCommand{BulletItem},
% \CodeCommand{SubBulletItem},
% etc.) do not carry across different paragraphs.
% \end{detail}

% \Gap
% \BulletItem
% Vertical Spacing Between Items
% \begin{detail}
% \SubBulletItem
% Use \CodeCommand{Gap} or \CodeCommand{GapNoBreak} to insert a small vertical space between items within the same section.
% \SubBulletItem
% Use \CodeCommand{BigGap} or \CodeCommand{BigGapNoBreak} to insert a bigger vertical space between items within the same section.
% \SubBulletItem
% The ``NoBreak'' versions prevent page breaking.
% \end{detail}

% \Gap
% \BulletItem
% Dates
% \begin{detail}
% \SubBulletItem
% Use
% \CodeCommand{DatestampYMD\{YYYY\}\{MM\}\{DD\}},
% \CodeCommand{DatestampYM\{YYYY\}\{MM\}}, and
% \CodeCommand{DatestampY\{YYYY\}}
% to specify dates.
% \SubBulletItem
% Change the date format option passed to the document class to adjust how dates are displayed throughout the document:
% MMMyyyy (``Dec~2010''),
% ddMMMyyyy (``31~Dec~2010''),
% MMMMyyyy (``December~2010''),
% ddMMMMyyyy (``31~December~2010''),
% yyyyMMdd (``2010-12-31''),
% yyyyMM (``2010-12''),
% yyyy (``2010'').
% \end{detail}

% %%%%%%%%%%%%%%%%%%%%%%%%%%%%%%%%%%%
% %% MULTILINGUAL UNICODE EXAMPLES %%
% %%%%%%%%%%%%%%%%%%%%%%%%%%%%%%%%%%%

% \section
% {Multilingual Unicode Examples}
% {Multilingual Unicode Examples}
% {PDF:MultilingualUnicodeExamples}

% \BulletItem
% Assortment of unicode characters from
% \href{http://www.ltg.ed.ac.uk/~richard/unicode-sample.html}
% {http://www.ltg.ed.ac.uk/{\TildeSymbol}richard/unicode-sample.html}

% \begin{detail}
% \SubItem
% \textbf{Latin Extended-A}
% Ā ā Ă ă Ą ą Ć ć Ĉ ĉ Ċ ċ Č č Ď ď Đ đ Ē ē Ĕ ĕ Ė ė Ę ę Ě ě Ĝ ĝ Ğ ğ Ġ ġ Ģ ģ Ĥ ĥ Ħ ħ Ĩ ĩ Ī ī Ĭ ĭ Į į İ ı IJ ij Ĵ ĵ
% \textbf{Latin Extended-B}
% ƀ Ɓ Ƃ ƃ Ƅ ƅ Ɔ Ƈ ƈ Ɖ Ɗ Ƌ ƌ ƍ Ǝ Ə Ɛ Ƒ ƒ Ɠ Ɣ ƕ Ɩ Ɨ Ƙ ƙ ƚ ƛ Ɯ Ɲ ƞ Ɵ Ơ ơ Ƣ ƣ Ƥ ƥ Ʀ Ƨ ƨ Ʃ ƪ ƫ Ƭ ƭ Ʈ Ư ư Ʊ Ʋ Ƴ ƴ Ƶ
% \textbf{Latin Extended Additional}
% Ḁ ḁ Ḃ ḃ Ḅ ḅ Ḇ ḇ Ḉ ḉ Ḋ ḋ Ḍ ḍ Ḏ ḏ Ḑ ḑ Ḓ ḓ Ḕ ḕ Ḗ ḗ Ḙ ḙ Ḛ ḛ Ḝ ḝ Ḟ ḟ Ḡ ḡ Ḣ ḣ Ḥ ḥ Ḧ ḧ Ḩ ḩ Ḫ ḫ Ḭ ḭ Ḯ ḯ Ḱ ḱ Ḳ ḳ Ḵ ḵ
% \textbf{Greek}
% ʹ ͵ ͺ ; ΄ ΅ Ά · Έ Ή Ί Ό Ύ Ώ ΐ Α Β Γ Δ Ε Ζ Η Θ Ι Κ Λ Μ Ν Ξ Ο Π Ρ Σ Τ Υ Φ Χ Ψ Ω Ϊ Ϋ ά έ ή ί ΰ α β γ δ ε ζ η θ
% \textbf{Cyrillic}
% Ё Ђ Ѓ Є Ѕ І Ї Ј Љ Њ Ћ Ќ Ў Џ А Б В Г Д Е Ж З И Й К Л М Н О П Р С Т У Ф Х Ц Ч Ш Щ Ъ Ы Ь Э Ю Я а б в г д е ж з
% \textbf{Hebrew}
% א ב ג ד ה ו ז ח ט י ך כ ל ם מ ן נ ס ע ף פ ץ צ ק ר ש ת װ ױ ײ ֝ ֞ ֟ ֠ ֡ ֣ ֤ ֥ ֦ ֧ ֨ ֩ ֪ ֫ ֬ ֭ ֮ ֯ ְ ֱ ֒ ֓ ֔
% \textbf{Armenian}
% {\UseSecondaryFont
% Ա Բ Գ Դ Ե Զ Է Ը Թ Ժ Ի Լ Խ Ծ Կ Հ Ձ Ղ Ճ Մ Յ Ն Շ Ո Չ Պ Ջ Ռ Ս Վ Տ Ր Ց Ւ Փ Ք Օ Ֆ ՙ ՚ ՛ ՜ ՝ ՞ ՟ ա բ գ դ ե զ}
% \textbf{Thai}
% {\UseSecondaryFont
% ก ข ฃ ค ฅ ฆ ง จ ฉ ช ซ ฌ ญ ฎ ฏ ฐ ฑ ฒ ณ ด ต ถ ท ธ น บ ป ผ ฝ พ ฟ ภ ม ย ร ฤ ล ฦ ว ศ ษ ส ห ฬ อ ฮ ฯ ะ ั า ำ ิ}
% \end{detail}

\end{body}

% %%%%%%%%%%%
% % CV NOTE %
% %%%%%%%%%%%


\end{document}
